\documentclass[11pt]{article}

\title{Clebsch-Gordan coefficients\footnote
{Copyright 2007 by Edmond Orignac.
This file is released under the terms of the GNU General Public License, version 2.}}

\begin{document}

\maketitle

\section{Wigner recoupling coefficients}

The Maxima script \texttt{clebsch\_gordan.mac} defines the $3j,6j$ and $9j$ 
 coefficients that are used in the theory of addition of angular momenta 
in quantum mechanics\cite{landau_mecaq,messiah_field_chapter}. 

% The Book of Messiah is available from Dover in an english translation.
% Angular momenta and Wigner coefficients can be found in Appendix C.  
% An online reference is:
% Weisstein, Eric W. "Wigner 9j-Symbol." From MathWorld--A Wolfram Web Resource.% http://mathworld.wolfram.com/Wigner9j-Symbol.html 

\subsection{Wigner $3j$ coefficients}

The Maxima function \texttt{wigner\_3j(j1,j2,m1,m2,j,m)} computes the $3j$ 
coefficient of Wigner.

The Wigner $3j$ coefficient appears in the addition of a pair of angular 
momenta in Quantum Mechanics. 
It is defined as\cite{landau_mecaq,messiah_field_chapter}:
\begin{equation}
  \label{eq:3j-def}
  \left(\begin{array}{ccc} j_1 & j_2 & j \\ m_1 & m_2 & m\end{array} \right) = (-1)^{j_1-j_2-m} \frac 1 {\sqrt{2j+1}} \langle j_1,m_1; j_2, m_2 | j,-m\rangle,   
\end{equation} 

where $ \langle j_1,m_1; j_2, m_2 | j,m\rangle$ is the 
Clebsch-Gordan coefficient. The Clebsch-Gordan coefficient is used to 
construct the state of total angular momentum $j$ and total projection 
of angular momentum $m$ as a linear combination of states of angular momenta 
$j_1$ and $j_2$ and respective projections $m_1$ and $m_2$.
One has:
\begin{equation}
  |j,m\rangle = \sum_{m_1,m_2} \langle j_1,m_1;j_2,m_2|j,m\rangle |j_1,m_1\rangle  |j_2,m_2\rangle  
\end{equation}
The advantage of working with the $3j$ coefficients instead of the 
Clebsch-Gordan coefficients is that the former are more symmetric\cite{landau_mecaq}. 
 

The $3j$ coefficient is computed by application of  
Eq. (27.9.1) p. 1006 of \cite{abramowitz_math_functions}.  


\subsection{Wigner $6j$ coefficients}

The Maxima function \texttt{wigner\_6j(j1,j2,j3,j4,j5,j6)} computes the $6j$ 
coefficient of Wigner.

The Wigner $6j$ coefficients appears in the addition of three angular momenta. 
When one is adding three angular momenta, one can form a first 
pair of angular momenta, add them together to form a new angular momentum
using the $3j$ coefficients, and add the resulting angular momenta with the
remaining angular momentum\cite{landau_mecaq,messiah_field_chapter}.
There are 3 different ways of grouping the angular momenta, which leads to 
different representations of the total angular momentum. 
The Wigner $6j$ coefficients are used to pass from one representation to the 
other. 

The notation for the $6j$ symbols is:
\begin{equation}
  \left\{\begin{array}{ccc}j_1 & j_2 & j_3 \\ j_4 & j_5 & j_6\end{array} \right\} 
\end{equation}

The $6j$ coefficient is computed by application of the formula  p. 513 Eq. (108,10) of \cite{landau_mecaq} or the equivalent formula  p. 915, Eq. (36) of \cite{messiah_field_chapter}. 


\subsection{Wigner $9j$ coefficients}

The function\texttt{wigner\_9j(a,b,c,d,e,f,h,i,j)} computes 
the $9j$ coefficient of Wigner. 

The $9j$ coefficients appears in the addition of four angular momenta. 
To add these angular momenta, one can first form two pairs of angular 
momenta and add them together to form the two resulting angular momenta 
and then add together the two resulting angular momenta. 
There are different ways to form the two pairs of angular momenta, and 
the $9j$ coefficient is used to transform from one representation to
the other\cite{landau_mecaq,messiah_field_chapter}. 

The notation for the $9j$ coefficient is:
\begin{equation}
  \left\{\begin{array}{cccc} a & b & c \\ d & e & f \\ h & i & j \end{array} \right\} 
\end{equation}

The $9j$ coefficient is computed by applying Eq. (41) p. 917, of \cite{messiah_field_chapter}. 


\section{Limitations}

The $3nj$ with $n\ge 4$ (addition of $n+1$ angular momenta) are not 
implemented. The theory of these coefficients 
can be found in the book Edmonds Angular momentum 
in quantum Mechanics (Princeton University Press). 

Various other coefficients can be defined that are related to the $3nj$ 
coefficients such as the Racah $W$ or $X$ coefficients\cite{messiah_field_chapter}. These coefficients are not implemented.

 
As the computation is done using exact formulas, it will break down if the
angular momenta that are entered are too large. For these cases, one should
implement recurrence formulas or use asymptotic expansions. 
 


\begin{thebibliography}{1}

\bibitem{abramowitz_math_functions}
{\sc Abramowitz, M., and Stegun, I.}
\newblock {\em Handbook of mathematical functions}.
\newblock Dover, New York, 1972.

\bibitem{landau_mecaq}
{\sc Landau, L.~D., and Lifshitz, E.~M.}
\newblock {\em Quantum Mechanics : non-relativistic theory}.
\newblock perg, New York, 1962.

\bibitem{messiah_field_chapter}
{\sc Messiah, A.}
\newblock {\em M\'ecanique Quantique}, vol.~2.
\newblock Dunod, Paris, 1995.

\end{thebibliography}


\end{document}
